% \iffalse meta-comment
%
% Copyright (C) 2021 by Robert McNees <rmcnees@luc.edu>, Leo C. Stein <leo.stein@gmail.com>
% ---------------------------------------------------------------------------
% This work may be distributed and/or modified under the
% conditions of the LaTeX Project Public License, either version 1.3
% of this license or (at your option) any later version.
% The latest version of this license is in
%   http://www.latex-project.org/lppl.txt
% and version 1.3 or later is part of all distributions of LaTeX
% version 2005/12/01 or later.
%
% This work has the LPPL maintenance status `maintained'.
%
% The Current Maintainers of this work are Robert McNees, Leo C. Stein.
%
% This work consists of the files graphpaper.dtx and graphpaper.ins
% and the derived filebase graphpaper.sty.
%
% \fi
%
% \iffalse
%<*driver>
\ProvidesFile{graphpaper.dtx}
%</driver>
%
%<*driver>
\documentclass{ltxdoc}
\usepackage[dvipsnames]{xcolor}
\usepackage[pattern=stdeight, textarea,
  majorcolor=cornflower!30, minorcolor=cornflower!10,
  geometry={margin=1in, left=2in}]{graphpaper}[2021/03/13]
\usepackage{hyperref}
\hypersetup{colorlinks,urlcolor=NavyBlue,citecolor=NavyBlue,linkcolor=NavyBlue,
  pdfauthor={Robert McNees and Leo C. Stein},
  pdftitle={The graphpaper package}}
\usepackage{fontawesome}
\definecolor{twitterblue}{RGB}{64,153,255}
\newcommand{\email}[1]{$\langle$\href{mailto:#1}{#1}$\rangle$}
\newcommand{\twitter}[1]{\href{https://twitter.com/#1}{\textcolor{twitterblue}{\faTwitter}\,\tt\textcolor{twitterblue}{@#1}}}
\usepackage{dtxdescribe}
\usepackage[columns=2]{idxlayout}
\EnableCrossrefs
\CodelineIndex
\RecordChanges
\begin{document}
  \DocInput{graphpaper.dtx}
  \PrintChanges
\end{document}
%</driver>
% \fi
%
% \CheckSum{388}
%
% \CharacterTable
%  {Upper-case    \A\B\C\D\E\F\G\H\I\J\K\L\M\N\O\P\Q\R\S\T\U\V\W\X\Y\Z
%   Lower-case    \a\b\c\d\e\f\g\h\i\j\k\l\m\n\o\p\q\r\s\t\u\v\w\x\y\z
%   Digits        \0\1\2\3\4\5\6\7\8\9
%   Exclamation   \!     Double quote  \"     Hash (number) \#
%   Dollar        \$     Percent       \%     Ampersand     \&
%   Acute accent  \'     Left paren    \(     Right paren   \)
%   Asterisk      \*     Plus          \+     Comma         \,
%   Minus         \-     Point         \.     Solidus       \/
%   Colon         \:     Semicolon     \;     Less than     \<
%   Equals        \=     Greater than  \>     Question mark \?
%   Commercial at \@     Left bracket  \[     Backslash     \\
%   Right bracket \]     Circumflex    \^     Underscore    \_
%   Grave accent  \`     Left brace    \{     Vertical bar  \|
%   Right brace   \}     Tilde         \~}
%
%
% \changes{v1.0.0}{2021/03/13}{Converted to DTX file}
%
% \DoNotIndex{\newcommand,\newenvironment}
%
% \GetFileInfo{graphpaper.sty}
% \title{The \textsf{graphpaper} package}
% \author{Robert McNees\footnote{\email{rmcnees@luc.edu}; \twitter{mcnees}}
%   \ and
%   Leo C.\ Stein\footnote{\email{leo.stein@gmail.com}; \twitter{duetosymmetry}}}
% \date{\fileversion~from \filedate}
%
% \maketitle
%
% \noindent Source repo:
% \url{https://github.com/mcnees/LaTeX-Graph-Paper}.
%
% \section{Introduction}
%
% Make your own quadrille, graph, hex, etc.\ paper! Uses the PGF/TikZ
% package for \LaTeX, which should be part of any modern \TeX{}
% installation. All colors and spacing are customizable.
% TODO Continue
%
% \section{Usage}
%
% \subsection{Options}
%
% Your graph paper is configured through a number of key/value options
% to the |\usepackage| command.  Let's go through these options.
%
% \DescribeObject{pattern=\marg{name}}\DescribeDefault{std} Which of
% the predefined patterns to use for the page or textarea background.
% The current list of pattern names is: |std|, |stdeight|, |majmin|,
% |dot|, |hex|, |tri|, |iso|, |lightcone|, |ruled|, |doubleruled|. We
% describe each of these patterns in Sec.~\ref{sec:patterns}.
% Patterns come with default page geometry (size and margins; see
% |geometry|), and default `fullness' (whether they fill the page or
% not; see options |fullpage| and |textarea|).
%
% \DescribeObject{colorset=\marg{name}}\DescribeDefault{std} Color
% presets.  Valid color preset names are: |std|, |precocious|,
% |brickred|, |engineer|, |plumpad|. A preset determines the
% |majorcolor|, |minorcolor|, and |bgcolor| all at once. But, you can
% start from a preset and then override some colors.
%
% \DescribeObject{majorcolor=\marg{color}} Override the preset
% ``major'' color. This can be a named color, or using the syntax from
% |xcolor| to mix colors together.
%
% \DescribeObject{minorcolor=\marg{color}} Override the preset
% ``minor'' color. As above.
%
% \DescribeObject{bgcolor=\marg{color}} Override the preset background
% color. As above.
%
% \DescribeObject{hexagonsize=\marg{length}}\DescribeDefault{0.1666in}
% Controls the size of hexagons for |pattern=hex|.
%
% \DescribeObject{trianglesize=\marg{length}}\DescribeDefault{0.25in}
% Controls the size of triangles for |pattern=tri| or
% |pattern=iso|.
%
% \DescribeObject{dotgridsize=\marg{length}}\DescribeDefault{0.1in}
% Controls the spacing of dots for |pattern=dot|.
%
% \DescribeObject{dotsize=\marg{length}}\DescribeDefault{.7pt}
% Controls the size of the dots themselves for |pattern=dot|.
%
% \DescribeObject{fullpage} Make the pattern fill the whole page.
%
% \DescribeObject{textarea} Make the pattern fill only the text area
% of the document. At most one of the |fullpage| or |textarea| can be
% specified. If one is specified, it will override the default
% `fullness' setting of the pattern.
%
% \DescribeObject{geometry=\marg{geometry spec}} Page geometry
% specification, using the syntax of the |geometry| package. This
% specification will override the pattern's default page geometry.
% However, if the |geometry| package was loaded before |graphpaper|,
% this option will be ignored.
%
% \subsection{Patterns}
% \label{sec:patterns}
%
% The current set of patterns:
%
% \DescribeObject{std} Quadrille, ten squares per inch.
%
% \DescribeObject{stdeight} Quadrille, eight squares per inch.
%
% \DescribeObject{majmin} Graph paper, eight squares per inch with a
% major grid every half-inch.
%
% \DescribeObject{dot} Grid of dots.  The dot spacing is set by
% |dotgridsize|, and the size of an individual dot is set by
% |dotsize|.
%
% \DescribeObject{hex} Grid of hexagons.  The hexagon size is set by
% |hexagonsize|.
%
% \DescribeObject{tri} Triangle grid.  The triangle size is set by
% |trianglesize|.
%
% \DescribeObject{iso} Isometric grid.  The triangle size is set by
% |trianglesize|.
%
% \DescribeObject{lightcone} A grid with light cones (45° lines) on a
% square grid.
%
% \DescribeObject{ruled} Ruled page with bold lines every 0.2in or
% 0.25in.
%
% \DescribeObject{doubleruled} Ruled page with bold lines every 0.25in
% and light lines every 0.125 in.
%
% \section{Examples}
%
% Let's say you want to use the |tri| pattern, which by default fills
% the page. But you want it to fill just the textarea of an A4 page
% with 2cm margins, and you want the triangles to be .75cm
% long. Finally, you like the colors of the engineer set, but want a
% white background. Then you would write:
% \begin{sourceverb}
%   \usepackage[pattern=tri,
%     trianglesize=0.75cm,
%     textarea,
%     colorset=engineer,
%     bgcolor=white,
%     geometry={a4paper, margin=2cm}]{graphpaper}
% \end{sourceverb}
%

%
% \StopEventually{}
%
% \section{Implementation}
%
% \iffalse
%<*package>
% \fi
%
%    \begin{macrocode}
\NeedsTeXFormat{LaTeX2e}[1994/06/01]
\ProvidesPackage{graphpaper}
    [2021/03/14 v1.0.0 Graph paper backgrounds]

\RequirePackage{xkeyval}
\RequirePackage{kvoptions}
\RequirePackage{xcolor}
\RequirePackage{tikz}
\usetikzlibrary{patterns.meta,calc}
\RequirePackage{tikzpagenodes}
%% everypage has been superseded -- try to use the new builtin
%% approach, but fall back to everypage-1x if needed
%% This code is roughly taken from the new everypage code
\@ifundefined{AddToHook}{%
  \RequirePackage{everypage-1x}
}{%
  \newcommand*{\AddEverypageHook}[1]{%
  \AddToHook{shipout/background}{\put(1in,-1in){#1}}}
}
\RequirePackage{pagecolor}

%%%%%%%%%%%%%%%%%%%%%%%%%%%%%%%%%%%%%%%%%%%%%%%%%%%%%%%%%%%%
%% Option parsing
%% Declare switches for processing the options.

\newif\ifGP@geometrypreviouslyloaded
\newif\ifGP@fullnessset
\newif\ifGP@fullpage
\newif\ifGP@textarea
\GP@geometrypreviouslyloadedfalse
\GP@fullnesssetfalse
\GP@fullpagefalse
\GP@textareafalse

\SetupKeyvalOptions{%
  family=GP,%
  prefix=GPOpt@%
}

\DeclareStringOption[std]{pattern}
\DeclareStringOption[std]{colorset}

\DeclareStringOption{majorcolor}
\DeclareStringOption{minorcolor}
\DeclareStringOption{bgcolor}

\DeclareStringOption[0.1666in]{hexagonsize}
\DeclareStringOption[0.25in]{trianglesize}
\DeclareStringOption[0.1in]{dotgridsize}
\DeclareStringOption[.7pt]{dotsize}

\DeclareVoidOption{fullpage}{\GP@fullpagetrue}
\DeclareVoidOption{textarea}{\GP@textareatrue}

\DeclareStringOption{geometry}

\ProcessKeyvalOptions*

%% Can only have one of fullpage or textarea
\ifGP@fullpage
  \ifGP@textarea
    \PackageError{graphpaper}{%
      Can not specify both fullpage and textarea, please remove one option}{}
    \fi
  \GP@fullnesssettrue
\fi

\ifGP@textarea
  \GP@fullnesssettrue
\fi

%% We keep track of this to know whether or not we would be overriding
%% a previously-set page geometry
\@ifpackageloaded{geometry}
  {\GP@geometrypreviouslyloadedtrue}
  {\GP@geometrypreviouslyloadedfalse%
    \PassOptionsToPackage{\GPOpt@geometry}{geometry}%
    \RequirePackage{geometry}%
  }

%%%%%%%%%%%%%%%%%%%%%%%%%%%%%%%%%%%%%%%%%%%%%%%%%%%%%%%%%%%%
%% Actual package code
%%%%%%%%%%%%%%%%%%%%%%%%%%%%%%%%%%%%%%%%%%%%%%%%%%%%%%%%%%%%

%%%%%%%%%%%%%%%%%%%%%%%%%%%%%%%%%%%%%%%%%%%%%%%%%%%%%%%%%%%%%%%%%%%%%%
%% Some nice colors.
%%%%%%%%%%%%%%%%%%%%%%%%%%%%%%%%%%%%%%%%%%%%%%%%%%%%%%%%%%%%%%%%%%%%%%
\definecolor{plum}{rgb}{0.36078, 0.20784, 0.4}
\definecolor{chameleon}{rgb}{0.30588, 0.60392, 0.023529}
\definecolor{cornflower}{rgb}{0.12549, 0.29020, 0.52941}
\definecolor{scarlet}{rgb}{0.8, 0, 0}
\definecolor{brick}{rgb}{0.64314, 0, 0}
\definecolor{sunrise}{rgb}{0.80784, 0.36078, 0}
\definecolor{rosiebg}{RGB}{250,247,232}
\definecolor{rosiegrid}{RGB}{186,137,113}

%%%%%%%%%%%%%%%%%%%%%%%%%%%%%%%%%%%%%%%%%%%%%%%%%%%%%%%%%%%%%%%%%%%%%%
%% The color to use for the null directions when drawing lightcones.
%%%%%%%%%%%%%%%%%%%%%%%%%%%%%%%%%%%%%%%%%%%%%%%%%%%%%%%%%%%%%%%%%%%%%%
\colorlet{lightlines}{scarlet!30}

%%%%%%%%%%%%%%%%%%%%%%%%%%%%%%%%%%%%%%%%%%%%%%%%%%%%%%%%%%%%%%%%%%%%%%
%% Pre-defined Color schemes
%%%%%%%%%%%%%%%%%%%%%%%%%%%%%%%%%%%%%%%%%%%%%%%%%%%%%%%%%%%%%%%%%%%%%%
%% Here are some pre-defined color schemes for the paper background
%% and the major and minor grid lines.  These are switched by using
%% the option colorset=<name>.  The allowed values for colorset are in
%% the list below.
\define@choicekey*{GP}{colorset}[\val\nr]%
  %% Allowed values for colorset:
  {std,precocious,brickred,engineer,plumpad}[std]{%
  \ifcase\nr\relax
    %% std
    \colorlet{minorcolor}{cornflower!30}
    \colorlet{majorcolor}{cornflower!50}
    \colorlet{bgcolor}{white}
  \or
    %% precocious
    \colorlet{minorcolor}{rosiegrid!50}
    \colorlet{majorcolor}{rosiegrid}
    \colorlet{bgcolor}{rosiebg}
  \or
    %% brickred
    \colorlet{minorcolor}{brick!35}
    \colorlet{majorcolor}{brick!60}
    \colorlet{bgcolor}{scarlet!8}
  \or
    %% engineer
    \colorlet{minorcolor}{chameleon!50}
    \colorlet{majorcolor}{chameleon!80}
    \colorlet{bgcolor}{chameleon!10}
  \or
    %% plumpad
    \colorlet{minorcolor}{cornflower!40}
    \colorlet{majorcolor}{cornflower!70}
    \colorlet{bgcolor}{plum!10}
  \fi
}

%% Get the specified color set from the options
\def\@setkeyhelper#1#2{%
  \setkeys{GP}{#2=#1}
}
\expandafter\@setkeyhelper\expandafter{\GPOpt@colorset}{colorset}

%% If the user further specified majorcolor, minorcolor, and/or
%% bgcolor, we now override the selected colorset
\ifx\GPOpt@majorcolor\@empty
\else
  \colorlet{majorcolor}{\GPOpt@majorcolor}
\fi
\ifx\GPOpt@minorcolor\@empty
\else
  \colorlet{minorcolor}{\GPOpt@minorcolor}
\fi
\ifx\GPOpt@bgcolor\@empty
\else
  \colorlet{bgcolor}{\GPOpt@bgcolor}
\fi

%%%%%%%%%%%%%%%%%%%%%%%%%%%%%%%%%%%%%%%%%%%%%%%%%%%%%%%%%%%%%%%%%%%%%%
%% This section sets up a routine for filling a shape with
%% hexagons. Uses code from:
%% http://tex.stackexchange.com/questions/6019/drawing-hexagons/6128#6128
%%%%%%%%%%%%%%%%%%%%%%%%%%%%%%%%%%%%%%%%%%%%%%%%%%%%%%%%%%%%%%%%%%%%%%

%% Hexagon size controlled by \GPOpt@hexagonsize
%% QUESTION: Should we just have 'majorsize' and 'minorsize' and use
%% that value here?

\pgfdeclarepatternformonly
  {hexagons}%% name
  {\pgfpointorigin}%% lower left
  {\pgfpoint{3*\GPOpt@hexagonsize}{0.866025*2*\GPOpt@hexagonsize}}
  {\pgfpoint{3*\GPOpt@hexagonsize}{0.866025*2*\GPOpt@hexagonsize}}
  {
    \pgfsetlinewidth{0.6pt}
    \pgftransformshift{\pgfpoint{0mm}{0.866025*\GPOpt@hexagonsize}}
    \pgfpathmoveto{\pgfpoint{0mm}{0mm}}
    \pgfpathlineto{\pgfpoint{0.5*\GPOpt@hexagonsize}{0mm}}
    \pgfpathlineto{\pgfpoint{\GPOpt@hexagonsize}{-0.866025*\GPOpt@hexagonsize}}
    \pgfpathlineto{\pgfpoint{2*\GPOpt@hexagonsize}{-0.866025*\GPOpt@hexagonsize}}
    \pgfpathlineto{\pgfpoint{2.5*\GPOpt@hexagonsize}{0mm}}
    \pgfpathlineto{\pgfpoint{3*\GPOpt@hexagonsize}{0mm}}
    \pgfpathmoveto{\pgfpoint{0.5*\GPOpt@hexagonsize}{0mm}}
    \pgfpathlineto{\pgfpoint{\GPOpt@hexagonsize}{0.866025*\GPOpt@hexagonsize}}
    \pgfpathlineto{\pgfpoint{2*\GPOpt@hexagonsize}{0.866025*\GPOpt@hexagonsize}}
    \pgfpathlineto{\pgfpoint{2.5*\GPOpt@hexagonsize}{0mm}}
    \pgfusepath{stroke}
  }

%%%%%%%%%%%%%%%%%%%%%%%%%%%%%%%%%%%%%%%%%%%%%%%%%%%%%%%%%%%%%%%%%%%%%%
%% This section sets up a routine for filling a shape with
%% triangles.
%%%%%%%%%%%%%%%%%%%%%%%%%%%%%%%%%%%%%%%%%%%%%%%%%%%%%%%%%%%%%%%%%%%%%%

%% Triangle size controlled by \GPOpt@trianglesize
%% QUESTION: Should we just have 'majorsize' and 'minorsize' and use
%% that value here?

\pgfdeclarepatternformonly
  %% Name of the pattern
  {triangles}
  %% Set the lower left corner of the pattern
  {\pgfpointorigin}
  %% Set the upper right corner of the pattern
  {\pgfpoint{\GPOpt@trianglesize}{2*0.8660254*\GPOpt@trianglesize}}
  %% Declare the size of the pattern blocks
  {\pgfpoint{\GPOpt@trianglesize}{2*0.8660254*\GPOpt@trianglesize}}
  %% Draw the pattern
  {
    \pgfsetlinewidth{0.6pt}
  \pgfpathmoveto{\pgfpoint{0mm}{0mm}}
  \pgfpathlineto{\pgfpoint{\GPOpt@trianglesize}{2*0.8660254*\GPOpt@trianglesize}}
  \pgfpathlineto{\pgfpoint{0mm}{2*0.8660254*\GPOpt@trianglesize}}
    \pgfpathmoveto{\pgfpoint{0mm}{0.8660254*\GPOpt@trianglesize}}
  \pgfpathlineto{\pgfpoint{\GPOpt@trianglesize}{0.8660254*\GPOpt@trianglesize}}
    \pgfpathmoveto{\pgfpoint{0mm}{2*0.8660254*\GPOpt@trianglesize}}
  \pgfpathlineto{\pgfpoint{\GPOpt@trianglesize}{0mm}}
  \pgfpathlineto{\pgfpoint{0mm}{0mm}}
    \pgfusepath{stroke}
  }

\pgfdeclarepatternformonly
  %% Name of the pattern
  {isometric}
  %% Set the lower left corner of the pattern
  {\pgfpointorigin}
  %% Set the upper right corner of the pattern
  {\pgfpoint{2*0.8660254*\GPOpt@trianglesize}{\GPOpt@trianglesize}}
  %% Declare the size of the pattern blocks
  {\pgfpoint{2*0.8660254*\GPOpt@trianglesize}{\GPOpt@trianglesize}}
  %% Draw the pattern
  {
    \pgfsetlinewidth{0.6pt}
  \pgfpathmoveto{\pgfpoint{0mm}{0mm}}
  \pgfpathlineto{\pgfpoint{2*0.8660254*\GPOpt@trianglesize}{\GPOpt@trianglesize}}
  \pgfpathlineto{\pgfpoint{2*0.8660254*\GPOpt@trianglesize}{0mm}}
  \pgfpathlineto{\pgfpoint{0mm}{\GPOpt@trianglesize}}
  \pgfpathlineto{\pgfpoint{0mm}{0mm}}
    \pgfpathmoveto{\pgfpoint{0.8660254*\GPOpt@trianglesize}{0mm}}
    \pgfpathlineto{\pgfpoint{0.8660254*\GPOpt@trianglesize}{\GPOpt@trianglesize}}
    \pgfusepath{stroke}
  }

%%%%%%%%%%%%%%%%%%%%%%%%%%%%%%%%%%%%%%%%%%%%%%%%%%%%%%%%%%%%%%%%%%%%%%
%% This section sets up a routine for filling the squares in a
%% grid with null lines.
%% %%%%%%%%%%%%%%%%%%%%%%%%%%%%%%%%%%%%%%%%%%%%%%%%%%%%%%%%%%%%%%%%%%%%
%% TODO Still can't figure out the correct pattern shift!!
%% TODO Make an option
\def\squaresize{0.25in}
\pgfkeys{
  /pgf/pattern keys/myshift/.store in=\myshift,
  /pgf/pattern keys/myshift/.initial={(0,0)},
}
\tikzdeclarepattern{
  name=lightcones,
  type=uncolored,
  parameters={\myshift},
  bounding box={(0,0) and (\squaresize,\squaresize)},
  tile size={(\squaresize, \squaresize)},
  tile transformation={
    shift=\myshift,
  },
  defaults={
    myshift/.store in=\myshift,myshift={(0,0)},
  },
  code={
    %% TODO Make the dashing an option
    \tikzset{lightlines/.style={line width=0.4pt,dash=on 0.05cm off 0.05cm phase 0.025cm}}
    \draw [lightlines] (0,0) -- (\squaresize,\squaresize);
    \draw [lightlines] (0,\squaresize) -- (\squaresize,0);
  },
}
%% \pgfdeclarepatternformonly
%%   {lightcones}% name
%%   {\pgfpointorigin}% lower left
%%   {\pgfpoint{\squaresize}{\squaresize}}%  upper right
%%   {\pgfpoint{\squaresize}{\squaresize}}%  tile size
%%   {% shape description
%%     \pgfsetlinewidth{0.4pt}
%%     %% TODO Make an option
%%     %Comment out this line for solid lines on light cones, instead of dashes.
%%     \pgfsetdash{{0.05cm}{0.05cm}}{0.025cm}
%%     \pgfpathmoveto{\pgfpoint{0in}{0in}}
%%     \pgfpathlineto{\pgfpoint{\squaresize}{\squaresize}}
%%     \pgfpathmoveto{\pgfpoint{0in}{\squaresize}}
%%     \pgfpathlineto{\pgfpoint{\squaresize}{0in}}
%%     \pgfusepath{stroke}
%%   }

%%%%%%%%%%%%%%%%%%%%%%%%%%%%%%%%%%%%%%%%%%%%%%%%%%%%%%%%%%%%%%%%%%%%%%
%% This section sets up a routine for filling a region with dots
%% Slightly modified version of code added by Leo
%% Stein (@duetosymmetry on Twitter).
%%%%%%%%%%%%%%%%%%%%%%%%%%%%%%%%%%%%%%%%%%%%%%%%%%%%%%%%%%%%%%%%%%%%%%
%% Dot grid size controlled by \GPOpt@dotgridsize
%% QUESTION: Should we just have 'majorsize' and 'minorsize' and use
%% that value here?

\pgfdeclarepatternformonly
  {dotgrid}%% name
  {\pgfpoint{-0.5*\GPOpt@dotgridsize}{-0.5*\GPOpt@dotgridsize}}%% lower left
  {\pgfpoint{0.5*\GPOpt@dotgridsize}{0.5*\GPOpt@dotgridsize}}%%  upper right
  {\pgfpoint{\GPOpt@dotgridsize}{\GPOpt@dotgridsize}}%%  tile size
  {%% shape description
    \pgfpathcircle{\pgfqpoint{0pt}{0pt}}{\GPOpt@dotsize}
    \pgfusepath{fill}
  }

%%%%%%%%%%%%%%%%%%%%%%%%%%%%%%%%%%%%%%%%%%%%%%%%%%%%%%%%%%%%%%%%%%%%%%
%% Begin pattern execution infrastructure
%%%%%%%%%%%%%%%%%%%%%%%%%%%%%%%%%%%%%%%%%%%%%%%%%%%%%%%%%%%%%%%%%%%%%%

%% This inner code will be set by the choicekey pattern=...
\newcommand{\GP@innerpatterncode}{}
%% This is the "outer" code to hook into every page
\newcommand{\GP@patterncode}{% No blank lines in this code!
\begin{tikzpicture}[remember picture, overlay]
%%
%% Change "thin" to "very thin" if the lines are too thick.
\tikzset{
  minorgrid/.style={minorcolor, thin},
  majorgrid/.style={majorcolor, thin},
}
\ifGP@fullpage%
\coordinate (a) at (current page.south west);
\coordinate (b) at (current page.north east);
\else%
\coordinate (a) at (current page text area.south west);
\coordinate (b) at (current page text area.north east);
\fi
%%
\GP@innerpatterncode%
%%
\end{tikzpicture}
}
%% Set the background color.
\AtBeginDocument{\pagecolor{bgcolor}}
%% Actually hook it in!
\AddEverypageHook{%
\GP@patterncode%
}

%%%%%%%%%%%%%%%%%%%%%%%%%%%%%%%%%%%%%%%%%%%%%%%%%%%%%%%%%%%%%%%%%%%%%%
%% Begin pattern definition code
%%%%%%%%%%%%%%%%%%%%%%%%%%%%%%%%%%%%%%%%%%%%%%%%%%%%%%%%%%%%%%%%%%%%%%

\define@boolkey{GP}{patterndefaultfullness}{}
\newcommand{\GP@patterndefaultgeometry}{}

%% Pattern-definer-helper
%% The interface is:
%% \GP@setpattern
%%   {<true for default fullpage, false for default textarea>}
%%   {<default geometry config>}
%%   {<contents of inner pattern code>}
\newcommand{\GP@setpattern}[3]{%
\setkeys{GP}{patterndefaultfullness=#1}
\renewcommand{\GP@patterndefaultgeometry}{#2}
\renewcommand{\GP@innerpatterncode}{#3}
}

\define@choicekey*{GP}{pattern}[\val\nr]%
  %% Allowed values for pattern:
  {std,stdeight,majmin,dot,hex,tri,iso,lightcone,ruled,doubleruled}{%
  \ifcase\nr\relax
    %% std
%%%%%%%%%%%%%%%%%%%%%%%%%%%%%%%%%%%%%%%%%%%%%%%%%%%%%%%%%%%%%%%%%%%%%%
%% Quadrille, ten squares per inch.
%% %%%%%%%%%%%%%%%%%%%%%%%%%%%%%%%%%%%%%%%%%%%%%%%%%%%%%%%%%%%%%%%%%%%%
    \GP@setpattern{false}{letterpaper, margin=0.2in}{%
%% Draw a grid with 10 squares per inch.
\draw[style=minorgrid, shift={(a)}] (0,0) grid [step=0.1in] (b);
%%
%% Draw a frame around the grid.
\draw[style=majorgrid] (a) rectangle (b);
    }
  \or
    %% stdeight
%%%%%%%%%%%%%%%%%%%%%%%%%%%%%%%%%%%%%%%%%%%%%%%%%%%%%%%%%%%%%%%%%%%%%%
%% Quadrille, eight squares per inch.
%%%%%%%%%%%%%%%%%%%%%%%%%%%%%%%%%%%%%%%%%%%%%%%%%%%%%%%%%%%%%%%%%%%%%%
    \GP@setpattern{false}{letterpaper, margin=0.1875in}{%
%% Draw a grid with 10 squares per inch.
\draw[style=minorgrid, shift={(a)}] (0,0) grid [step=0.125in] (b);
%%
%% Draw a frame around the grid.
\draw[style=majorgrid] (a) rectangle (b);
    }
  \or
    %% majmin
%%%%%%%%%%%%%%%%%%%%%%%%%%%%%%%%%%%%%%%%%%%%%%%%%%%%%%%%%%%%%%%%%%%%%%
%% Graph paper, eight squares per inch with a major grid
%% every half-inch.
%%%%%%%%%%%%%%%%%%%%%%%%%%%%%%%%%%%%%%%%%%%%%%%%%%%%%%%%%%%%%%%%%%%%%%
    \GP@setpattern{false}{letterpaper, margin=0.25in}{%
%% Draw a grid with 10 squares per inch.
\draw[style=minorgrid, shift={(a)}] (0,0) grid [step=0.125in] (b);
%%
\draw[style=majorgrid, shift={(a)}] (0,0) grid [step=0.5in]   (b);
%%
%% Draw a frame around the grid.
\draw[style=majorgrid] (a) rectangle (b);
    }
  \or
    %% dot
%%%%%%%%%%%%%%%%%%%%%%%%%%%%%%%%%%%%%%%%%%%%%%%%%%%%%%%%%%%%%%%%%%%%%%
%% Dot grid
%% Slightly modified version of code added by Leo
%% Stein (@duetosymmetry).
%%%%%%%%%%%%%%%%%%%%%%%%%%%%%%%%%%%%%%%%%%%%%%%%%%%%%%%%%%%%%%%%%%%%%%
    \GP@setpattern{true}{}{%
    \fill [pattern=dotgrid,pattern color=minorcolor] (a) rectangle (b);
    }
  \or
    %% hex
%%%%%%%%%%%%%%%%%%%%%%%%%%%%%%%%%%%%%%%%%%%%%%%%%%%%%%%%%%%%%%%%%%%%%%
%% Hex grid
%%%%%%%%%%%%%%%%%%%%%%%%%%%%%%%%%%%%%%%%%%%%%%%%%%%%%%%%%%%%%%%%%%%%%%
    \GP@setpattern{true}{}{%
    \fill [pattern=hexagons,pattern color=minorcolor] (a) rectangle (b);
    }
  \or
    %% tri
%%%%%%%%%%%%%%%%%%%%%%%%%%%%%%%%%%%%%%%%%%%%%%%%%%%%%%%%%%%%%%%%%%%%%%
%% Triangle grid, adjust triangle size in the preamble
%%%%%%%%%%%%%%%%%%%%%%%%%%%%%%%%%%%%%%%%%%%%%%%%%%%%%%%%%%%%%%%%%%%%%%
    \GP@setpattern{true}{}{%
      \fill [pattern=triangles,pattern color=minorcolor] (a) rectangle (b);
    }
  \or
    %% iso
%%%%%%%%%%%%%%%%%%%%%%%%%%%%%%%%%%%%%%%%%%%%%%%%%%%%%%%%%%%%%%%%%%%%%%
%% Isometric grid
%%%%%%%%%%%%%%%%%%%%%%%%%%%%%%%%%%%%%%%%%%%%%%%%%%%%%%%%%%%%%%%%%%%%%%
    \GP@setpattern{true}{}{%
      \fill [pattern=isometric, pattern color=minorcolor] (a) rectangle (b);
    }
  \or
    %% lightcone
%%%%%%%%%%%%%%%%%%%%%%%%%%%%%%%%%%%%%%%%%%%%%%%%%%%%%%%%%%%%%%%%%%%%%%
%% A grid with light cones.
%%%%%%%%%%%%%%%%%%%%%%%%%%%%%%%%%%%%%%%%%%%%%%%%%%%%%%%%%%%%%%%%%%%%%%
    \GP@setpattern{false}{letterpaper, margin=.125in}{%
%% Draw a grid with 4 squares per inch.
\draw[style=minorgrid, shift={(a)}] (0,0) coordinate grid [step=0.25in] (b);
%%
%% Draw a border around the grid.
\draw[style=majorgrid, pattern={lightcones[myshift={(a)}]}, pattern color=lightlines] (a) rectangle (b);
    }
  \or
    %% ruled
%%%%%%%%%%%%%%%%%%%%%%%%%%%%%%%%%%%%%%%%%%%%%%%%%%%%%%%%%%%%%%%%%%%%%%
%% Ruled page with bold lines every 0.2in or 0.25in TODO make ystep an option
%%%%%%%%%%%%%%%%%%%%%%%%%%%%%%%%%%%%%%%%%%%%%%%%%%%%%%%%%%%%%%%%%%%%%%
    \GP@setpattern{false}{letterpaper, body={8in,10.8in}}{%
%% Draw a ruled page with lines every 0.2in
\draw[style=majorgrid, shift={(a)}] (0,0) grid [ystep=0.2in, xstep=\paperwidth] (b);
    }
  \or
    %% doubleruled
%%%%%%%%%%%%%%%%%%%%%%%%%%%%%%%%%%%%%%%%%%%%%%%%%%%%%%%%%%%%%%%%%%%%%%
%% Ruled page with bold lines every 0.25in and light lines
%% every 0.125 in.
%%%%%%%%%%%%%%%%%%%%%%%%%%%%%%%%%%%%%%%%%%%%%%%%%%%%%%%%%%%%%%%%%%%%%%
    \GP@setpattern{false}{letterpaper, margin=.25in}{%
%% Draw a ruled pattern with thin lines every 0.125 in and bold lines every 0.25 in.
\draw[style=minorgrid, shift={(a)}] (0,0) grid [ystep=0.125in, xstep=\paperwidth] (b);
%%
\draw[style=majorgrid, shift={(a)}] (0,0) grid [ystep=0.25in, xstep=\paperwidth] (b);
%%
%% Draw a frame around the grid.
\draw[style=majorgrid] (a) rectangle (b);
    }
  \fi
}

%% Use the passed package option to set the above key
\expandafter\@setkeyhelper\expandafter{\GPOpt@pattern}{pattern}

%% Determine whether or not to (re)set fullpage vs textarea
\ifGP@fullnessset
%% Respect their choice
\else
  %% Reset the value of \GP@fullpage based on the pattern's default
  %% There's probably a more idiomatic way to do this but I can't
  %% figure it out
  \ifKV@GP@patterndefaultfullness
    \GP@fullpagetrue
  \else
    \GP@fullpagefalse
  \fi
\fi

%% Determine whether or not to fiddle with the page geometry
\ifGP@geometrypreviouslyloaded
%% Respect their previous choice
\PackageWarning{graphpaper}{'geometry' package was previously loaded, will not use pattern defaults.}
\else
  %% Use the pattern's defaults,
  \expandafter\geometry\expandafter{\GP@patterndefaultgeometry}
  %% And then override with any more specific settings passed by the user
  \expandafter\geometry\expandafter{\GPOpt@geometry}
\fi

\endinput
%    \end{macrocode}
%
% \iffalse
%</package>
% \fi
%
% \Finale
\endinput
